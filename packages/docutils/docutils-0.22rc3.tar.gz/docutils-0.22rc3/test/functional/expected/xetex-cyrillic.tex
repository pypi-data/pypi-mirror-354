\documentclass[a4paper,russian]{article}
% generated by Docutils <https://docutils.sourceforge.io/>
% rubber: set program xelatex
\usepackage{fontspec}
% \defaultfontfeatures{Scale=MatchLowercase}
% straight double quotes (defined in T1 but missing in TU):
\ifdefined \UnicodeEncodingName
  \DeclareTextCommand{\textquotedbl}{\UnicodeEncodingName}{%
    {\addfontfeatures{RawFeature=-tlig,Mapping=}\char34}}%
\fi
\usepackage{polyglossia}
\setdefaultlanguage{russian}
\setotherlanguages{english}
\setcounter{secnumdepth}{0}

%%% Custom LaTeX preamble
% Linux Libertine (free, wide coverage, not only for Linux)
\setmainfont{Linux Libertine O}
\setsansfont{Linux Biolinum O}
\setmonofont[HyphenChar=None,Scale=MatchLowercase]{DejaVu Sans Mono}

%%% User specified packages and stylesheets

%%% Fallback definitions for Docutils-specific commands
% hyperlinks:
\ifdefined\hypersetup
\else
  \usepackage[hyperfootnotes=false,
              colorlinks=true,linkcolor=blue,urlcolor=blue]{hyperref}
  \usepackage{bookmark}
  \urlstyle{same} % normal text font (alternatives: tt, rm, sf)
\fi
\hypersetup{
  pdflang={ru},
}


%%% Body
\begin{document}


\section{Заголовок%
  \label{section-1}%
}

первый пример: «Здравствуй, мир!»


\section{Title%
  \label{title}%
}

\foreignlanguage{english}{first example: “Hello world”.}


\section{Примечания%
  \label{section-2}%
}

Этот документ испытивает изображение LaTeXом кириллических букв.

\foreignlanguage{english}{This document tests the rendering of Latin and Cyrillic characters by the
LaTeX and XeTeX writers. Check the compiled PDF for garbage characters in
text and bookmarks.}

\end{document}
